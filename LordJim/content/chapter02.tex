\chapter{}
 After two years of training he went to sea, and entering the regions so well known to his imagination, found them strangely barren of adventure. He made many voyages. He knew the magic monotony of existence between sky and water: he had to bear the criticism of men, the exactions of the sea, and the prosaic severity of the daily task that gives bread—but whose only reward is in the perfect love of the work. This reward eluded him. Yet he could not go back, because there is nothing more enticing, disenchanting, and enslaving than the life at sea. Besides, his prospects were good. He was gentlemanly, steady, tractable, with a thorough knowledge of his duties; and in time, when yet very young, he became chief mate of a fine ship, without ever having been tested by those events of the sea that show in the light of day the inner worth of a man, the edge of his temper, and the fibre of his stuff; that reveal the quality of his resistance and the secret truth of his pretences, not only to others but also to himself.

Only once in all that time he had again a glimpse of the earnestness in the anger of the sea. That truth is not so often made apparent as people might think. There are many shades in the danger of adventures and gales, and it is only now and then that there appears on the face of facts a sinister violence of intention—that indefinable something which forces it upon the mind and the heart of a man, that this complication of accidents or these elemental furies are coming at him with a purpose of malice, with a strength beyond control, with an unbridled cruelty that means to tear out of him his hope and his fear, the pain of his fatigue and his longing for rest: which means to smash, to destroy, to annihilate all he has seen, known, loved, enjoyed, or hated; all that is priceless and necessary—the sunshine, the memories, the future; which means to sweep the whole precious world utterly away from his sight by the simple and appalling act of taking his life.

Jim, disabled by a falling spar at the beginning of a week of which his Scottish captain used to say afterwards, ‘Man! it’s a pairfect meeracle to me how she lived through it!’ spent many days stretched on his back, dazed, battered, hopeless, and tormented as if at the bottom of an abyss of unrest. He did not care what the end would be, and in his lucid moments overvalued his indifference. The danger, when not seen, has the imperfect vagueness of human thought. The fear grows shadowy; and Imagination, the enemy of men, the father of all terrors, unstimulated, sinks to rest in the dullness of exhausted emotion. Jim saw nothing but the disorder of his tossed cabin. He lay there battened down in the midst of a small devastation, and felt secretly glad he had not to go on deck. But now and again an uncontrollable rush of anguish would grip him bodily, make him gasp and writhe under the blankets, and then the unintelligent brutality of an existence liable to the agony of such sensations filled him with a despairing desire to escape at any cost. Then fine weather returned, and he thought no more about It.

His lameness, however, persisted, and when the ship arrived at an Eastern port he had to go to the hospital. His recovery was slow, and he was left behind.

There were only two other patients in the white men’s ward: the purser of a gunboat, who had broken his leg falling down a hatchway; and a kind of railway contractor from a neighbouring province, afflicted by some mysterious tropical disease, who held the doctor for an ass, and indulged in secret debaucheries of patent medicine which his Tamil servant used to smuggle in with unwearied devotion. They told each other the story of their lives, played cards a little, or, yawning and in pyjamas, lounged through the day in easy-chairs without saying a word. The hospital stood on a hill, and a gentle breeze entering through the windows, always flung wide open, brought into the bare room the softness of the sky, the languor of the earth, the bewitching breath of the Eastern waters. There were perfumes in it, suggestions of infinite repose, the gift of endless dreams. Jim looked every day over the thickets of gardens, beyond the roofs of the town, over the fronds of palms growing on the shore, at that roadstead which is a thoroughfare to the East,—at the roadstead dotted by garlanded islets, lighted by festal sunshine, its ships like toys, its brilliant activity resembling a holiday pageant, with the eternal serenity of the Eastern sky overhead and the smiling peace of the Eastern seas possessing the space as far as the horizon.

Directly he could walk without a stick, he descended into the town to look for some opportunity to get home. Nothing offered just then, and, while waiting, he associated naturally with the men of his calling in the port. These were of two kinds. Some, very few and seen there but seldom, led mysterious lives, had preserved an undefaced energy with the temper of buccaneers and the eyes of dreamers. They appeared to live in a crazy maze of plans, hopes, dangers, enterprises, ahead of civilisation, in the dark places of the sea; and their death was the only event of their fantastic existence that seemed to have a reasonable certitude of achievement. The majority were men who, like himself, thrown there by some accident, had remained as officers of country ships. They had now a horror of the home service, with its harder conditions, severer view of duty, and the hazard of stormy oceans. They were attuned to the eternal peace of Eastern sky and sea. They loved short passages, good deck-chairs, large native crews, and the distinction of being white. They shuddered at the thought of hard work, and led precariously easy lives, always on the verge of dismissal, always on the verge of engagement, serving Chinamen, Arabs, half-castes—would have served the devil himself had he made it easy enough. They talked everlastingly of turns of luck: how So-and-so got charge of a boat on the coast of China—a soft thing; how this one had an easy billet in Japan somewhere, and that one was doing well in the Siamese navy; and in all they said—in their actions, in their looks, in their persons—could be detected the soft spot, the place of decay, the determination to lounge safely through existence.

To Jim that gossiping crowd, viewed as seamen, seemed at first more unsubstantial than so many shadows. But at length he found a fascination in the sight of those men, in their appearance of doing so well on such a small allowance of danger and toil. In time, beside the original disdain there grew up slowly another sentiment; and suddenly, giving up the idea of going home, he took a berth as chief mate of the Patna.

The Patna was a local steamer as old as the hills, lean like a greyhound, and eaten up with rust worse than a condemned water-tank. She was owned by a Chinaman, chartered by an Arab, and commanded by a sort of renegade New South Wales German, very anxious to curse publicly his native country, but who, apparently on the strength of Bismarck’s victorious policy, brutalised all those he was not afraid of, and wore a ‘blood-and-iron’ air,’ combined with a purple nose and a red moustache. After she had been painted outside and whitewashed inside, eight hundred pilgrims (more or less) were driven on board of her as she lay with steam up alongside a wooden jetty.

They streamed aboard over three gangways, they streamed in urged by faith and the hope of paradise, they streamed in with a continuous tramp and shuffle of bare feet, without a word, a murmur, or a look back; and when clear of confining rails spread on all sides over the deck, flowed forward and aft, overflowed down the yawning hatchways, filled the inner recesses of the ship—like water filling a cistern, like water flowing into crevices and crannies, like water rising silently even with the rim. Eight hundred men and women with faith and hopes, with affections and memories, they had collected there, coming from north and south and from the outskirts of the East, after treading the jungle paths, descending the rivers, coasting in praus along the shallows, crossing in small canoes from island to island, passing through suffering, meeting strange sights, beset by strange fears, upheld by one desire. They came from solitary huts in the wilderness, from populous campongs, from villages by the sea. At the call of an idea they had left their forests, their clearings, the protection of their rulers, their prosperity, their poverty, the surroundings of their youth and the graves of their fathers. They came covered with dust, with sweat, with grime, with rags—the strong men at the head of family parties, the lean old men pressing forward without hope of return; young boys with fearless eyes glancing curiously, shy little girls with tumbled long hair; the timid women muffled up and clasping to their breasts, wrapped in loose ends of soiled head-cloths, their sleeping babies, the unconscious pilgrims of an exacting belief.

‘Look at dese cattle,’ said the German skipper to his new chief mate.

An Arab, the leader of that pious voyage, came last. He walked slowly aboard, handsome and grave in his white gown and large turban. A string of servants followed, loaded with his luggage; the Patna cast off and backed away from the wharf.

She was headed between two small islets, crossed obliquely the anchoring-ground of sailing-ships, swung through half a circle in the shadow of a hill, then ranged close to a ledge of foaming reefs. The Arab, standing up aft, recited aloud the prayer of travellers by sea. He invoked the favour of the Most High upon that journey, implored His blessing on men’s toil and on the secret purposes of their hearts; the steamer pounded in the dusk the calm water of the Strait; and far astern of the pilgrim ship a screw-pile lighthouse, planted by unbelievers on a treacherous shoal, seemed to wink at her its eye of flame, as if in derision of her errand of faith.

She cleared the Strait, crossed the bay, continued on her way through the ‘One-degree’ passage. She held on straight for the Red Sea under a serene sky, under a sky scorching and unclouded, enveloped in a fulgor of sunshine that killed all thought, oppressed the heart, withered all impulses of strength and energy. And under the sinister splendour of that sky the sea, blue and profound, remained still, without a stir, without a ripple, without a wrinkle—viscous, stagnant, dead. The Patna, with a slight hiss, passed over that plain, luminous and smooth, unrolled a black ribbon of smoke across the sky, left behind her on the water a white ribbon of foam that vanished at once, like the phantom of a track drawn upon a lifeless sea by the phantom of a steamer.

Every morning the sun, as if keeping pace in his revolutions with the progress of the pilgrimage, emerged with a silent burst of light exactly at the same distance astern of the ship, caught up with her at noon, pouring the concentrated fire of his rays on the pious purposes of the men, glided past on his descent, and sank mysteriously into the sea evening after evening, preserving the same distance ahead of her advancing bows. The five whites on board lived amidships, isolated from the human cargo. The awnings covered the deck with a white roof from stem to stern, and a faint hum, a low murmur of sad voices, alone revealed the presence of a crowd of people upon the great blaze of the ocean. Such were the days, still, hot, heavy, disappearing one by one into the past, as if falling into an abyss for ever open in the wake of the ship; and the ship, lonely under a wisp of smoke, held on her steadfast way black and smouldering in a luminous immensity, as if scorched by a flame flicked at her from a heaven without pity.

The nights descended on her like a benediction. 